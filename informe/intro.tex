\section{Introduction}\label{intro}

El an\'alisis de la performance de una aerol\'inea es importante para
distintos agentes: la aerovl\'inea en s\'i, para analizar aspectos a 
mejorar; los clientes, para elegir la mejor opci\'on a la hora de realizar
un vuelo; la competencia, para tomar mejores decisiones de mercado;
los gobiernos, para analizar la aerol\'inea de bandera y su 
competitividad, etc\'etera. El estudio de los retrasos de los vuelos
es fundamental: vuelos retrasados pueden causar inconvenientes en
la organizaci\'on de los clientes, en sus combinaciones de vuelos.
Retrasos excesivos pueden ser causas de resarcimients econ\'omicos
e influir negativamente en la imagen de la empresa.

Los algoritmos de aprendizaje autom\'atico nos permiten realizar
predicciones de distintas variables en base a registros previos.
En este caso, contamos con datos del \textit{U.S. Departmen
of Transportation}\footnote{https://www.transtats.bts.gov/Fields.asp?Table\_ID=236},
utilizados en una competencia de visualizaci\'on
de datos de la \textit{American Statistical Asociation}\footnote{
http://stat-computing.org/dataexpo/2009/}, que nos proveen
de registros de vuelos en Estados Unidos de 1994 a 2008, con
informaci\'on sobre la fecha, el origen y el destino, tiempos
estimados y reales de despegue y aterrizaje, entre otros.

El an\'alisis de series temporales tiene amplia difusi\'on en la
literatura, con aplicaciones importantes en muchas \'areas del
conocimiento. En este trabajo, vamos a enfocarnos en un modelo de regresi\'on
lineal para aprender sobre el cuerpo de datos y realizar un pron\'ostico
a futuro.





This short note provides a guide to using the ENDM macro package for
preparing papers for publication in your conference \emph{Proceedings}.
The \emph{Proceedings} may be printed and hard copies distributed to
participants at the meeting; this is an option conference organizers
may choose to exercise.  The \emph{Proceedings} also will be part of
a volume in the series \emph{Electronic Notes in Discrete Mathematics}
(ENDM), which is published under the auspices of Elsevier B.~V., the
publishers of \emph{Discrete Mathematics} and \emph{Discrete Applied
Mathematics}. The ENDM home page can be found at
\href{http://www.elsevier.com/locate/endm}
{\texttt{http://www.elsevier.com/locate/endm}}

The ENDM macro package consists of two files:
\begin{itemize}
\item \texttt{endm.cls}. This is the basic style file.
\item \texttt{endmmacro.sty}. A macro file containing the definitions
of some of the theorem-like environments and a few other things.
\end{itemize}

The formatting these style files impose should \emph{not} be altered.
The reason for using them is to attain a uniform format for all papers
in the \emph{Proceedings} of which your paper is a part.

Additional macro files can be added using \verb+\usepackage{...}+.
The file \texttt{endmmacro.sty} \emph{must} be included in the
list, as is done at the start of the source file for these
instructions.

The ENDM package requires a relatively up-to-date \LaTeX\ system in
order to be successfully used. This is reflected in two other packages
that are called by endm.cls, which must be available on your machine.
These are:
\begin{itemize}
\item The \texttt{hyperref} package. This package allows the use of
hyperlinks in files prepared using \LaTeX 2e, one of the main features
of Adobe's Acrobat$^{\copyright}$ Reader software. Be sure that you
have at least version 6.69d of this package.
\item The \texttt{ifpdf} package. This is used by hyperref to
differentiate between the use of pdf\LaTeX\ and \LaTeX 2e, followed
by dvips and then ps2pdf.
\end{itemize}

The file \texttt{instraut.pdf} contains information about the use of
\LaTeX to prepare files for online publication by Elsevier. This file
refers to the older version of \LaTeX\ that is no longer suppported,
and that is inadequate for preparing \texttt{.pdf} files for online
publication. Reading this file should answer most of the basic
questions about \LaTeX\ that might arise.
