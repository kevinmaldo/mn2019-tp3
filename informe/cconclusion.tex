\section{Conclusi\'on}
En este trabajo pudimos apreciar las dificultades del problema de intentar predecir los
retrasos en los vuelos.
Observamos que hay factores que incrementan la posibilidad de que un vuelo se retrase,
como por ejemplo el dia de la semana o eventos estacionales, pero a\'un con estos es
dif\'icil poder tener certeza en las prediciones obtenidas, lo cual se vi\'o reflejado en
los valores que obtuvimos en m\'etricas que usamos, en particular la \testit{recall}: detectamos
un bajo porcentaje de verdaderos positivos.

La dificultad viene asociada a la complejidad inherente del problema: los retrasos en los vuelos dependen
de muchos m\'as factores que los solamente temporales peri\'odicos o de la aerol\'inea/aeropuerto. Distintos
eventos que no consideramos en el modelo, como el clima, contribuyen a retrasos en vuelos que son m\'as
dif\'iciles de predecir con la informaci\'on del dataset que estudiamos.

Mucho m\'as dif\'icil a\'un de predecir es el precio de las acciones de las aerol\'ineas, sujeto a
especulaciones financieras que muchas veces no est\'an directamente relacionadas con el estado
actual de la aerol\'inea sino con proyecciones a futuro, por ejemplo.

\subsection{Trabajo futuro}\label{sec:futurework}
A futuro se puede pensar en un modelo con m\'as features del dataset, o de otros datasets (por ejemplo,
uno que incluya datos del clima de cada regi\'on, o features autoregresivos).
Tambi\'en, se pueden aplicar otras t\'ecnicas comunes
en el \'area de procesamiento de se\~nales como la remoci\'on de outliers, suavizaci\'on de la se\~nal,
filtros por frecuencia (IIR o FIR), etc\'etera.

Adem\'as, se pueden pensar otros modelos, como los conocidos ARIMA, o aquellos basados en redes neuronales,
que sean m\'as robustos para la predicci\'on.

En cuanto al segundo eje de estudio relacionado con los precios de las acciones de las aerol\'ineas, se pueden
pensar qu\'e otros features tenemos a disposici\'on para intentar modelar a la bolsa de valores,
con las t\'ecnicas de la econometr\'ia a disposici\'on.

La implementaci\'on de cuadrados m\'inimos es perfectible, en cuanto a optimizaci\'on de memoria y uso de CPU.


\iffalse

Realizando una evaluación final sobre el trabajo, nos quedamos con una sensación ambigua.
A través del primer eje, pudimos ver las dificultades que tenía la predicción a gran escala,
pues al querer focalizar en una época del año o en cierto aeropuerto, lo que en un primer
momento parecía un método aceptable se transformó en un sistema de malas predicciones. A su vez,
este eje nos mostró los problemas de evaluar el error a través de ECM, ya que al trabajar con valores
de magnitud como lo son los delays en minutos, un error importante en una estimación generaría que el
valor medio crezca de forma desmesurada.

Sin embargo, al crear el segundo eje sobre una escala mucho más pequeña, con menos datos y una
hipótesis clara, nos tropezamos con resultados poco conclusivos, métodos cuyo nivel de predicción
eran bastante limitados y la incapacidad de definir la significancia de las variables, tanto del
método como de los datos provistos. Esto simplemente nos confundío más aún, pero repasando el trabajo
y pensando en frío sobre ambos ejes, llegamos a algunas conclusiones respecto al método en sí.

Si bien no obtuvimos buenos resultados al aplicar una hipótesis sobre el segundo eje, esta claro que
hubo una gran significancia de los primeros resultados vistos sobre el aeropuerto de Denver.
Sin
embargo, el condicionante fue una mala hipótesis basada en un eje que no habíamos evaluado
anteriormente (el delay aéreo). Probablemente, de haber generado una relación mayor entre lo
visto en el primer eje y lo propuesto en el segundo, habríamos alcanzado resultados mucho más vistosos,
producto de que contábamos con una serie de datos que nos guiarían durante la experimentación.

Así, aún cuando obtuvimos datos interesantes en el primer eje respecto a un caso particular,
los
mismos se vieron opacados al darle a la investigación un camino equivocado. De este modo, concluímos
que la idea de comenzar de forma general para luego buscar información particular fue correcta, pero
no así el planteo que le dimos al segundo eje, pues cualquier hipótesis con un respaldo pobre siempre
tendrá una mayor probabilidad de estar equivocada.

\fi