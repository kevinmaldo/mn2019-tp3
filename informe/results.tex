\section{Resultados}\label{sec:resultados}

\subsection{Features}\label{subsec:features}

Realizamos un \textit{scatter plot} de los datos, exponemos los resultados en la figura \ref{fig:trending}.

\begin{figure}[hbtp]
  \centering
  \includegraphics[width=\textwidth]{plots/linear_trending.png}
  \caption{\textit{Detrending} de los datos}
  \label{fig:trending}
\end{figure}

Realizamos un ajuste lineal de los datos, los que nos da un nuevo dataset con media cero (proceso conocido
como \textit{detrending}).

Luego, realizamos gr\'aficos que nos muestren c\'omo se comportan los retrasos dentro de los meses de un a\~no, los
d\'ias de un mes y los d\'ias de una semana. Exponemos los datos en las figuras \ref{fig:within_year}, \ref{fig:within_month}
y \ref{fig:within_week}.

\begin{figure}[hbtp]
  \centering
  \includegraphics[width=0.7\textwidth]{plots/within_year.png}
  \caption{Retraso medio por mes}
  \label{fig:within_year}
\end{figure}

\begin{figure}[hbtp]
  \centering
  \includegraphics[width=0.7\textwidth]{plots/within_month.png}
  \caption{Retraso medio por d\'ia del mes}
  \label{fig:within_month}
\end{figure}

\begin{figure}[hbtp]
  \centering
  \includegraphics[width=0.7\textwidth]{plots/within_week.png}
  \caption{Retraso medio por d\'ia de la semana}
  \label{fig:within_week}
\end{figure}

Las figuras muestran un comportamiento peri\'odico de los retrasos. Es m\'as claro el de la figura \ref{fig:within_year}
que tiene elevaciones en los meses diciembre/enero y junio/julio, que es esperable pues son
meses de vacaciones. Tambi\'en se ve en la figura \ref{fig:within_week} un aumento en los d\'ias mi\'ercoles/jueves
y domingo/lunes. El comportamiento parece ser menos predecible para la figura \ref{fig:within_month}.

A modo de ejemplo, se muestran los m\'as altos scores
(explicados en la secci\'on \ref{subsec:scores}) del a\~no 2008
en las figuras \ref{fig:carrier_scores} y \ref{fig:airport_scores}.

\begin{figure}[hbtp]
  \centering
  \includegraphics[width=0.6\textwidth, height=2in]{plots/carrier_scores_2008.png}
  \caption{Primeros puntajes de aerol\'ineas}
  \label{fig:carrier_scores}
\end{figure}

\begin{figure}[hbtp]
  \centering
  \includegraphics[width=0.7\textwidth, height=2in]{plots/airport_scores_2008.png}
  \caption{Primeros puntajes de aeropuertos (origen y destino)}
  \label{fig:airport_scores}
\end{figure}

Sobre la base de este an\'alisis se escogieron los features descritos en la secci\'on \ref{subsec:extraccion}.
Se ajust\'o un modelo de cuadrados m\'inimos (secci\'on \ref{subsec:prediction}).
En la figura \ref{fig:example_fit_prediction} se muestra el resultado del modelo con datos de entrenamiento
entre 2002 y 2006 y datos de test a partir de 2006.

\begin{figure}[hbtp]
  \centering
  \includegraphics[width=\textwidth, height=3in]{plots/example_fit_and_prediction.png}
  \caption{Ejemplo de predicci\'on. La l\'inea punteada delimita el fin de los datos de
  entrenamiento y el comienzo de los datos de test}
  \label{fig:example_fit_prediction}
\end{figure}

Nos planteamos la pregunta: \textquestiondown Aprende el algoritmo efectivamente de los datos? \textquestiondown Mejora la predicci\'on si
entrenamos con m\'as datos? Para responder a esta pregunta, ejecutamos el modelo entrenando con una cantidad
creciente de a\~nos, desde uno a todos menos uno, y testeamos contra los dem\'as a\~nos. Exponemos
los resultados de las m\'etricas en la tabla:

\begin{center}
 \resizebox{\textwidth}{!}{%
  \begin{tabular}{||c || c | c | c | c | c | c | c||}
 \hline
 Training years & Delay RMSE & Delay NRMSE & RMSE & Accuracy & Precision & Recall & Balanced Accuracy \\ [1ex]
 \hline\hline
 1 & 14.22 & 0.24 & 0.53 & 0.72 & 0.21 & 0 & 0.5 \\
 \hline
 2 & 14.13 & 0.24 & 0.54 & 0.71 & 0.46 & 0 & 0.5 \\
 \hline
 3 & 14.25 & 0.24 & 0.55 & 0.7 & 0.47 & 0.02 & 0.51 \\
 \hline
 4 & 14.68 & 0.24 & 0.56 & 0.69 & 0.47 & 0.02 & 0.50 \\
 \hline
 5 & 14.9 & 0.25 & 0.56 & 0.68 & 0.5 & 0.03 & 0.51 \\
 \hline
 6 & 14.85 & 0.25 & 0.56 & 0.69 & 0.49 & 0.04 & 0.51 \\ [1ex]
 \hline
\end{tabular}}
\end{center}