\section{Desarrollo}\label{sec:desarrollo}

\subsection{Implementaci\'on y ejecuci\'on}\label{sec:ejecucion}
El m\'etodo de cuadrados m\'inimos fue implementado en \codeword{C++} utilizando
la librer\'ia de c\'alculo num\'erico \codeword{Eigen} para la descomposici\'on SVD,
los experimentos fueron realizados en \codeword{Python} utilizando las librer\'ias
de c\'alculo num\'erico \codeword{numpy} y \codeword{scipy}, y
las librer\'ias de visualizaci\'on \codeword{matplotlib} y \codeword{seaborn}.
Todo se ejecut\'o en una m\'aquina con procesador \codeword{Intel(R) Core(TM) i7-6700K CPU @ 4.00GHz}
y \codeword{32 GB} de memoria RAM.

\subsection{Datos}\label{subsec:datos}
Los datos pertenecen al \textit{U.S. Departmen of
Transportation}\footnote{https://www.transtats.bts.gov/Fields.asp?Table\_ID=236},
utilizados en una competencia de visualizaci\'on
de datos de la \textit{American Statistical Association}\footnote{
http://stat-computing.org/dataexpo/2009/}.
Dado que los datos son susceptibles a problemas en Estados Unidos, en este trabajo
estudiamos los datos a partir de 2002 (para evitar el ruido generado por el atentado a las torres
gemelas) y antes de septiembre de 2008 (para no considerar los problemas de la recesi\'on de 2008).
Adem\'as, contamos con datos de los precios de las acciones de algunas aerol\'ineas, para
buscar correlaci\'on entre estos y los datos del dataset de vuelos. Estos fueron extra\'idos de
\codeword{Yahoo! Finance}\footnote{https://finance.yahoo.com/} y se encuentran en
\codeword{carrier_stock_prices/}.

\subsection{Transformaci\'on}\label{subsec:transformacion}
Al leer los datos, consideramos el retraso de un vuelo como: 0 si el vuelo est\'a adelantado o no est\'a
atrasado, o el m\'aximo entre el retraso de despegue y el retraso de aterrizaje (en minutos) con un
valor m\'aximo de 60 minutos. Estas decisiones fueron basadas en la mejor performance del algoritmo.

\subsection{Extracci\'on de features}\label{subsec:extraccion}
La extracci\'on de features se realiz\'o de manera iterativa: se plante\'o una hip\'otesis (por ejemplo:
"Los datos tienen una tendencia lineal creciente"), se ajust\'o una funci\'on acorde a la hip\'otesis
(por ejemplo: una funci\'on lineal), se elimin\'o esa componente de los datos (por ejemplo: restarle a los datos
el resultado de la funci\'on lineal) y se procedi\'o a estudiar la siguiente hip\'otesis.
Los features considerados fueron: tendencia lineal al crecimiento, periodicidad con distintas frecuencias,
pico de retraso para d\'ias cercanos al 31/12, retraso medio de aerol\'ineas y aeropuertos
(ver secci\'on \ref{subsec:scores}), d\'ia de la semana.
.

\subsection{Scores}\label{subsec:scores}
Le asignamos a cada aerol\'inea un puntaje por a\~no que es el estimado de retraso de la aerol\'inea ese a\~no,
 y a cada aeropuerto dos puntajes por a\~no, que son los estimados de retraso de ese aeropuerto
como origen o como destino de vuelo. Los puntajes est\'an entre 0 y 30 minutos como m\'aximo.

\subsection{Evaluaci\'on}\label{subsec:evaluacion}
Para la evaluaci\'on del algoritmo se utiliz\'o un conjunto de datos distinto al de entrenamiento,
se convirti\'o a los tiempos de retrasos en booleanos: 0 si el retraso es menor que 15 minutos, 1 si el retraso
es mayor o igual que 15 minutos, y sobre ese arreglo de booleanos se reportaron las m\'etricas:
\begin{itemize}
 \item Delay RMSE: RMSE de los tiempos de retraso de los vuelos, en minutos
 \item Delay NRMSE: RMSE de los tiempos de retraso de los vuelos, normalizado
 \item RMSE: RMSE del arreglo de booleanos
 \item Accuracy, precision, recall, balanced accuracy: del arreglo de booleanos
\end{itemize}